\chapter{Pruebas con usuarios}

No solo realizar los test para comprobar el correcto funcionamiento del código es importante, también lo es analizar la impresión que tienen los usuarios de nuestro producto. Por ello, se ha decidido llevar a cabo una evaluación de la usabilidad del Bot por medio de un test de usabilidad \textbf{SUS} (System Usability Scale).

A pesar de ser muy sencillo de realizar, a lo largo del tiempo, diferentes pruebas y test han llegado a la conclusión de que los resultas que se obtienen suelen ser muy confiables u acertados. Esta es una de las razones principales para llevar a cabo un medición de la usabilidad a través de la experiencia del usuario.

Este cuestionario está formado por 10 preguntas, las cuales se evaluarán de 1 a 5, donde 1 significa \textit{Total desacuerdo} y 5 \textit{Total acuerdo}. Las preguntas son las siguientes:

\begin{enumerate}
	\item Creo que usaría esta aplicación frecuentemente.
	\item Encuentro esta aplicación innecesariamente compleja.
	\item Creo que la aplicación fue fácil de usar.
	\item Creo que necesitaría ayuda de una persona con conocimientos técnicos para usar esta aplicación.
	\item Las funciones de esta aplicación están bien integradas.
	\item Creo que la aplicación es muy inconsistente.
	\item Imagino que la mayoría de la gente aprendería a usar esta aplicación.
	\item Encuentro que la aplicación es muy difícil de usar.
	\item Me siento confiado al usar esta aplicación.
	\item Necesité aprender muchas cosas antes de ser capaz de usar esta aplicación.
\end{enumerate}

Para obtener los resultados, vamos a sumar los resultados promedio obtenidos de los cuestionarios realizados a los usuarios, considerando lo siguiente: las preguntas impares (1,3,5,7 y 9) tomarán el valor asignado por el usuario, y se le restará 1. Para las preguntas pares (2,4,6,8,10), será de 5 menos el valor asignado por nuestros entrevistados. Una vez obtenido el número final, se lo multiplica por 2,5.

Para este cuestionario se han selecionado 10 usuarios entre los que se encuentran 2 personas ancianas, 2 adultos, 2 personas mayores de 30 años y 6 usuarios jóvenes, 3 con dispositivos Android y 3 con dispositivos iOS.

Los resultados de este cuestionario podremos verlos en la siguiente tabla:

\begin{table}[H]
	\begin{center}
		\begin{tabular}{| c | c | c | c | c | c | c | c | c | c | c |}
			\hline
			
			\textbf{P1} & \textbf{P2} & \textbf{P3} & \textbf{P4} & \textbf{P5} & \textbf{P6} & \textbf{P7} & \textbf{P8} & \textbf{P9} & \textbf{P10} & \textbf{Puntuación total} \\
			4 & 1 & 5 & 2 & 5 & 2 & 5 & 1 & 4 & 1 & 90\\
			5 & 1 & 5 & 1 & 5 & 1 & 5 & 1 & 4 & 1 & 97,5\\
			4 & 1 & 5 & 1 & 5 & 1 & 5 & 1 & 5 & 1 & 97,5\\
			3 & 1 & 5 & 1 & 4 & 2 & 5 & 1 & 5 & 1 & 90\\
			4 & 1 & 5 & 1 & 5 & 1 & 5 & 1 & 5 & 1 & 97,5\\
			4 & 1 & 5 & 1 & 5 & 1 & 5 & 1 & 5 & 1 & 97,5\\
			5 & 1 & 5 & 1 & 5 & 1 & 4 & 1 & 5 & 1 & 97,5\\
			5 & 1 & 5 & 1 & 4 & 1 & 5 & 1 & 5 & 1 & 97,5\\
			5 & 1 & 5 & 1 & 5 & 1 & 5 & 1 & 5 & 1 & 100\\
			5 & 1 & 5 & 1 & 5 & 1 & 5 & 1 & 5 & 1 & 100\\
			 &  &  &  &  &  &  &  &  & \textbf{Total} & \textbf{96,5}\\ \hline
		\end{tabular}
		\caption{Resultados cuestionario SUS.}
	\end{center}
\end{table}

Como se puede apreciar se ha obtenido una puntuación bastante alta. Dependiendo de la puntuación obtenida determinaríamos la usabilidad del Bot teniendo en cuenta el siguiente criterio:

\begin{itemize}
	\item \textbf{Puntuacion <= 51:} se debe considerar encarecidamente mejorar la usabilidad de la aplicación.
	\item \textbf{Puntuación aprox. 68} está bien pero pueden mejorarse aun algunos aspectos.
	\item \textbf{Puntuación >= 80:} la usabilidad de la aplicación, a los usuarios le gusta y la recomendarán.
\end{itemize}

Como se puede der en la tabla anterior, el resultado del cuestionario es de \textbf{96,5}, una puntuación muy alta y que nos reporta que la usabilidad de nuestra aplicación es muy elevada. Es cierto que en algunos puntos aún es mejorable, como por ejemplo en el uso de esta aplicación de manera asidua por los usuarios.

Con esto lo que queremos mostrar es que el proyecto va en una dirección correcta aunque haya aún puntos en los que se puede mejorar. Al ser un Bot y ser su medio de publicidad el propio boca a boca de la gente, el reportar una puntuación como la obtenida nos indica que lo que se pretende se puede cumplir y que el Bot cada vez será usado por más gente.