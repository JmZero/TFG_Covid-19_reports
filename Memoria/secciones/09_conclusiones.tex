\chapter{Conclusiones}

Este proyecto se ha desarrollado en plena segunda ola del Covid-19, donde hemos vuelto a vivir con las restricciones y los contagios han aumentado, así como los fallecidos. Los datos que suelen mostrarnos en las noticias y a veces son insuficientes. Por ello \textbf{Covid-19 Reports} se presenta como una gran fuente de datos para todas las personas que están preocupadas por la evolución de la pandemia o si solo quieren informarse de algún dato concreto.

Mucha de esta información ya se encuentra en diferente medios, pero a veces una misma persona tiene que consultar varias web de noticias para consultar todos los datos. Por ello \textbf{Covid-19 Reports} intentará mostrarle el mayor nº de información posible a los usuarios, presentándose como una potencial herramienta de información.

A lo largo del desarrollo de este proyecto y al trabajar con los datos he podido comprobar como ha ido aumentando la pandemia. En el momento de inicio, aún se permitía la movilidad de la gente y en el momento se su finalización, la provincia de Granada tiene una de las mayores tasas de contagios de Andalucía y existen limitaciones de apertura de comercios, el famoso "toque de queda" y de movilidad ya no solo entre provincias, si no entre diferentes municipios. Como he dicho, al "trabajar con los datos en la mano" te das cuenta de la gravedad de la situación por la que estamos pasando, aunque haya gente a la que parece no importarle.

Para concluir, como reflexión personal, entiendo que mucha gente quiere mantener su vida de antes, quedar con sus amigos, pareja o familia y no es algo malo. Ya hemos visto a lo largo del verano que se puede controlar en mayor o menor medida este virus, por ello no hay que confiarse y poner siempre el máximo nº de medios posibles. Como he dicho, lo entiendo, pero no comparto comportamientos vergonzosos como los que de viven a día de hoy, ya que yo soy el primero al que le gustaría poder cenar junto a su familia en un restaurante o tomarme algo con mis amigos en un bar, pero todos debemos ser conscientes de la situación que se está viviendo hoy en día.